\chapter{Implementation}
This chapter contain the technical specification of the navigation system and landing path system, in addition to the payload installed in the X8 and both nests. The main focus of this thesis is the implementation of the landing path system and the navigation system, including user interface for both system in Neptus. A simplified model of the DUNE system is shown in figure \ref{fig:DuneSystem}, where the task LandingPlan creates the landing path, RTKGPS dispatch the \gls{rtk-gps} solution, Navigation decide the navigation source, Path Control creates desired control input to the low level controllers and Ardupilot which communicates with the Pixhawk.
\newpage
\begin{figure}[H]
	\centering
		\includegraphics[width=1\textwidth]{figs/DUNESystem.png}
		\caption{A simplified figure of the Dune auto land system}
		\label{fig:DuneSystem}
\end{figure}
\section{Landing path}
The landing path system is design to start generation of a landing path when the \gls{imc} message LandingPlanGeneration message is dispatched in the DUNE system. The \gls{imc} message was created to ease the configuration of the landing path, and includes option on how the path should be created. Among the parametrisation options there is the possibility to manually decide the rotation direction of both the start and finish circle in the approach path. This option allow the user to create a desired path, which not necessary is the shortest path. The behaviour parameters in the LandingPlanGeneration message is given in table \ref{Tb:DubinConfig}. In addition the user can specify that the \gls{uav} should wait in a loiter manoeuvre at the end of the approach path before continuing towards the landing path. This is useful when performing dynamical landing where the position of the landing point is not fixed. In the case of performing a dynamic landing the land path system will only create the approach path, and not the landing path. This was found out to be a preferable solution since a dynamical landing require a feedback loop to correct the desired path, which is currently not included in the landing path system. A solution for performing a dynamical landing is currently research by fellow Master students where the multi-copters is used to catch the \gls{uav}, where this landing system is used to create a approach path to ready the \gls{uav} for a dynamic landing.

From Neptus the plug-in LandmapLayer, which is an altered version of Neptus plug-in developed in thesis \citep{Froelich}. Alteration in the plug-in include new parameters, the inclusion of the \gls{imc} message LandingPlangeneration and the abillity to manually write the global position coordinates of the net.
\begin{table}
\centering
\begin{tabular}{| p{2.7cm} | | p{6cm} |}
\hline
\textbf{Parameter name} 							& \textbf{Action} \\ \hline
 Automatic (boolean)								& If true a standard path where the shortest Dubins path is chosen. Otherwise a user specific path is chosen \\ \hline
Start circle turning counter clockwise (boolean)	& If true the start arc is created such that the turning direction is counter clockwise. Otherwise clockwise. Require Automatic==false \\ \hline
Finish circle turning counter clockwise (boolean)	& If true the finish arc is created such that the turning direction is counter clockwise. Otherwise clockwise. Require Automatic==false \\ \hline
Wait at loiter (boolean)							& If true a unlimited loiter is included in the path before the path continue with the path along the virtual runway. \\ \hline

\end{tabular}
\caption{Landing path behaviour setting in LandingPlanGeneration}
\label{Tb:DubinConfig}
\end{table}
\section{Navigation system}
The navigation system is control by a state machine \ref{S:NavState}, which is used to control the content of the output \gls{imc} messages EstimatedState and NavSources. Depending on which state the navigation system is in the \gls{imc} EstimatedState message will either have position solution form the \gls{rtk-gps} system or the external navigation system. During a short loss of the RTK the external navigation position is compensated with the average difference between the RTK solution and the external navigation solution.
\subsection{RTK-GPS system}
The \gls{rtk-gps} solution is dispatched from the DUNE task RTKGPS, however before the message is accepted by the Navigation task the message must include a valid base station position. The base station position is not included in the output message from RTKlib, which demand the base station position to be calculated locally at the base station as a standalone \gls{gnss} receiver. For this purpose the DUNE task BasestationFix is used to lock the current position of the base station, which result in the base station position being transmitted to the RTKGPS task.
The navigation system require to now the reference position of the base station in order to use the \gls{rtk-gps} solution. However the base station position is currently not part of the output message from rtkrcv. This is resolved by allowing the base station to calculate it's own position as a standalone \gls{gps}. The \gls{gps} position is transmitted to a local Dune task on the base station, where the operator can decide when the base station can be considered as fixed. When the base station is considered fixed the position is sent to the X8, where it's included in the \gls{rtk-gps} solution message.
\subsubsection{Nest system}
A nest system is a stationary unit with the sole purpose of providing it's position to the rest of the Dune System. As part of the navigation system the base station is defined as a nest, where the \gls{gps} position is sent to the \gls{rtk-gps} system when fixed.
An other nest has been created to obtain the \gls{gps} position of the stationary net. The net nest is configured as a rover in \gls{rtk-gps} configuration, such that the position relative to the base station is in the same frame as the X8.
\subsection{Operator interface}
The state of the navigation system is monitored though a interface in Neptus. The interface indicate which source the Dune system is using for state information. The interfaced apply a color code to indicate which source is currently in use in addition to all sensor system that are available, as seen in table \ref{Tb:Color Code}.
\begin{table}[H]
\begin{center}
    \begin{tabular}{ | l | l |}
    \hline
    \textbf{Color} & \textbf{Description} \\ \hline
    White & Not available \\ \hline
    Yellow & Available, but not in use \\ \hline
    Green & Available, and in use \\ \hline
    \end{tabular}
\end{center}
\caption{Net approach parameters }
\label{Tb:Color Code}
\end{table}