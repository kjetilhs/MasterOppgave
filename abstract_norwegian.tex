\pagestyle{empty}
\renewcommand{\abstractname}{Sammendrag}
\begin{abstract}
\noindent Denne avhandlingen presenterer ett bane- og navigasjonssystem, som brukes i et autonomt nett landingssystem for en fast-vinge \acrfull{uav}. En landings bane til en \gls{uav} kan konstrueres som en rett linjet bane, men for at en \gls{uav} skal kunne følge landings banen må \gls{uav}en være i en posisjon hvor det finnes en mulig bane til landings banens start posisjonen. Dette motiverer utviklingen av en innflyvingsbane logikk mot landings banen sin inngangs posisjon fra hvilken som helst innledende startposisjon i luftrommet.

I tillegg til et bane genererende system trenger \gls{uav}en ett robust høy nøyaktig navigasjonssystem. Dette gjøres med å bruke \acrfull{rtk-gnss}, som kan gi centimeter nøyaktighet. En ulempe med et \gls{rtk-gnss} er at det kan miste lås på satellitter som fører til tap av funksjonalitet. I dette arbeidet kompenseres dette ved å benytte et sekundær \acrfull{gnss} system. For å håndtere tap av \gls{rtk-gnss} er et robust \gls{rtk-gnss} system foreslått, hvor den tidligere gyldige \gls{rtk-gnss} posisjon løsninger er kombinert med det sekundære \gls{gnss} systemet, for å bli brukt til å kompensere det eksterne navigasjon systemet. Denne kompensatoren er utformet slik at det eksterne navigasjonssystemet får samme nøyaktighets nivå som \gls{rtk-gnss} i en kort periode, inntil enten \gls{rtk-gnss} kommer tilbake eller blir frakoblet. Med denne kompensatoren blir navigasjonssystemet til \gls{uav}en robust om korte tap av \gls{rtk-gnss}, og tilgjenneligheten til \gls{rtk-gnss} er utvidet.

Både eksperimentell testing og \acrfull{sil} verifikasjon har blitt utført for å teste \gls{uav}ens evne til å lande i et stasjonært nett plassert på en tilfeldig posisjon. For å bedre navigasjon og ytelse, har en mobil sensor enhet blitt benyttet for å få posisjonen til nettet. Denne sensoren enheten ble benyttet under testing, hvor den viste seg å gi det tiltenkte bidraget.
\end{abstract}