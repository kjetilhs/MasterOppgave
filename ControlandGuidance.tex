\chapter{Guidance and control system}\label{AP:ControlGuidanceSystem}
The autonomous landing system is design to be independent from any types of guidance and control system. However this hold only true when the guidance and control system is implemented in the DUNE environment due to the requirement of an accurate navigation system. 
\section{Lateral controller}
The lateral controller used in the autonomous landing system is based on the paper \citep{fortuna2015cascaded}, which is a line of sight path following and sliding mode controller. The controller is designed to converge to the straight line between two waypoints, with the ability to compensate the wind disturbance in order to stay on the path. Depending on the line of sight of the controller, the controller either react aggressive towards wind disturbance or allow the wind to guide the \gls{uav}. The lookahead distance is a constant parameter, which can be configured from either the configuration file in DUNE or Neptus. The lateral controller is implemented in the DUNE task LOSnSMC, however a slightly altered version was used in the autonomous landing field experiment, which was the DUNE task LOSnSMCuser. The configuration parameter used in the autonomous landing field test are given in table \ref{AP:TB:LOSnSMCuser} 

\begin{table}[H]
\centering
\begin{tabular}{| l | l |}
\hline
\textbf{Parameter}	&	\textbf{Value} \\ \hline
Lookahead           &                    $50.0$ \\ \hline
Rho                 &                    $1.0$ \\ \hline
Lambda				&                    $0.35$ \\ \hline
Kd					&                    $1.5$ \\ \hline
Bandwidth			&                    $3.0$ \\ \hline
Roll Time Const		&                    $0.5$ \\ \hline
Maximum Bank		&                    $40.0$ \\ \hline
\end{tabular}
\caption{LOSnSMCuser configuration parameters used in the hardware setup}
\label{AP:TB:LOSnSMCuser}
\end{table}
\section{Longitudinal controller}
The longitudinal control system used in the autonomous landing system is designed and implemented as part of the master thesis \citep{Sigurd}. The controller is design to create glide slopes between waypoint, where a 3rd low pass reference model is used make the glide slope between two way point smooth enough to become flyable. The reference model create outputs a desired hight which lay above the desired path due to lag introduced from the reference model. When the desired height glide slope will reach a constant height is dependent on the time constant used in the reference model, in addition to the time of arrival factor in DUNE. The longitudinal control system is implemented in the DUNE tasks Longitudinal and HeightGlideslope, with the configuration parameter given in table \ref{AP:TB:Longitudianl} and \ref{AP:TB:HeightGlideslope} for the Longitudinal and HeightGlideslope respectfully.
\begin{table}[H]
\centering
\begin{tabular}{| l | l |}
\hline
\textbf{Parameter}	&	\textbf{Value} \\ \hline
Throttle Proportional gain          &    $0.3$ \\ \hline
Throttle Integrator gain            &    $0.1$ \\ \hline
Throttle Proportional height gain   &    $0.3$ \\ \hline
Gamma Proportional gain             &    $1.0$ \\ \hline
Trim throttle                       &    $44.0$ \\ \hline
\end{tabular}
\caption{Longitudinal configuration parameters used in the hardware setup}
\label{AP:TB:Longitudianl}
\end{table}

\begin{table}[H]
\centering
\begin{tabular}{| l | l |}
\hline
\textbf{Parameter}	&	\textbf{Value} \\ \hline
LOS Proportional gain up        &        $0.9$ \\ \hline
LOS Integral gain up            &        $0.1$ \\ \hline
LOS Radius up                   &        $14$ \\ \hline
LOS Proportional gain down       &       $0.9$ \\ \hline
LOS Integral gain down          &        $0.1$ \\ \hline
LOS Radius down                  &       $14$ \\ \hline
LOS Proportional gain line      &        $1.0$ \\ \hline
LOS Integral gain line          &        $0.02$ \\ \hline
LOS Radius line                &         $25$ \\ \hline
Time constant refmodelZ         &        $1.0$ \\ \hline
Time constant refmodelGamma     &        $1.0$ \\ \hline
Use reference model             &        True \\ \hline
Height bandwidth               &        $10$ \\ \hline
Vertical Rate maximum gain      &        $0.3$ \\ \hline
\end{tabular}
\caption{HeightGlideslope configuration parameters used in the hardware setup}
\label{AP:TB:HeightGlideslope}
\end{table}