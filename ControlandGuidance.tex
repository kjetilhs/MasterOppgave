\chapter{Guidance and control system}\label{AP:ControlGuidanceSystem}
The autonomous landing system is design to be independent from any types of guidance and control system. However this hold only true when the guidance and control system is implemented in the DUNE environment due to the requirement of an accurate navigation system. The control system is separated into two groups, high level and low level control. The low level control is the control loops for the actuators, which are only controlled by ardupilot. The high level controllers depend on the the configuration of Ardupilot, which is listed in table \ref{tb:ArduPilotMode}.
\begin{table}
\centering
\begin{tabular}{| p{3cm} | p{5cm}|}
\hline
\textbf{Mode}	&	\textbf{Description} \\ \hline
Guide			& Ardupilot set-point guidance and control system 												\\ \hline
FBWB			& DUNE lateral controller with desired height controlled in a set-point controller in Ardupilot \\ \hline
FBWA			& DUNE lateral and longitudianl controller, where control input is sent directly to the low level controllers in Ardupilot 																						\\ \hline
\end{tabular}
\caption{Guidance and control modes in autopilot viable for the landing system}
\label{tb:ArduPilotMode}
\end{table}

\section{Lateral controller}
The lateral controller used in the autonomous landing system is based on the paper \citep{fortuna2015cascaded}, which is a s

\begin{table}
\centering
\begin{tabular}{| l | l |}
\hline
\textbf{Parameter}	&	\textbf{Value} \\ \hline
Lookahead           &                    $50.0$ \\ \hline
Rho                 &                    $1.0$ \\ \hline
Lambda				&                    $0.35$ \\ \hline
Kd					&                    $1.5$ \\ \hline
Bandwidth			&                    $3.0$ \\ \hline
Roll Time Const		&                    $0.5$ \\ \hline
Maximum Bank		&                    $40.0$ \\ \hline

\end{tabular}
\end{table}
\section{Longitudinal controller}
The longitudinal controller is based on the paper \citep{you2012guidance}

Need to refer to the height smoothing filter with advantage and disadvantage.

\begin{table}
\centering
\begin{tabular}{| l | l |}
\hline
\textbf{Parameter}	&	\textbf{Value} \\ \hline
Throttle Proportional gain          &    $0.3$ \\ \hline
Throttle Integrator gain            &    $0.1$ \\ \hline
Throttle Proportional height gain   &    $0.3$ \\ \hline
Gamma Proportional gain             &    $1.0$ \\ \hline
Trim throttle                       &    $44.0$ \\ \hline
\end{tabular}
\end{table}

\begin{table}
\centering
\begin{tabular}{| l | l |}
\hline
\textbf{Parameter}	&	\textbf{Value} \\ \hline
LOS Proportional gain up        &        $0.9$ \\ \hline
LOS Integral gain up            &        $0.1$ \\ \hline
LOS Radius up                   &        $14$ \\ \hline
LOS Proportional gain down       &       $0.9$ \\ \hline
LOS Integral gain down          &        $0.1$ \\ \hline
LOS Radius down                  &       $14$ \\ \hline
LOS Proportional gain line      &        $1.0$ \\ \hline
LOS Integral gain line          &        $0.02$ \\ \hline
LOS Radius line                &         $25$ \\ \hline
Time constant refmodelZ         &        $1.0$ \\ \hline
Time constant refmodelGamma     &        $1.0$ \\ \hline
Use reference model             &        True \\ \hline
Height bandwidth               &        $10$ \\ \hline
Vertical Rate maximum gain      &        $0.3$ \\ \hline
\end{tabular}
\end{table}