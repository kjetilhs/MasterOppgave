\chapter{Applied software and hardware}
This chapter contain the software that is used for developing and testing the autonomous landing system for \gls{uav}. The software that is mainly used in the uav system is based on a open-source software toolchain developed by the Underwater System and Technology Laboratory (LSTS). The toolchain supports air and ocean vehicle systems. The different components in the toolchain is IMC, DUNE, NEPTUS and Glued, which will be presented in later in this chapter. 

The rtkgps solution in the system is calculated in the open-source software RtkLib \citep{takasu2009development}. The desceription of the program is given in section \ref{s:Rtklib}.

The low level control system in the uav is controlled by Ardupilot, which is a 
\section{LSTS toolchain}
The software that the system is based on was developed by the Underwater Systems and Technology Laboratory (LSTS), which is called the LSTS toolchain \citep{pinto2013lsts}. The toolchain was developed for support of networked heterogeneous air and ocean vehicle systems over wireless network. The toolchain contain four different modules, namely \gls{imc}, DUNE, NEPTUS and Glued.
\subsection{IMC}
\gls{imc} \citep{martins2009imc} is design to enable interconnections between systems of vehicles, sensors and human operators, which enable the pursuit of common goal by cooperatively exchange real-time information about the environment and updated objectives. The message protocol is oriented around the message, which abstracts hardware and communication heterogeneity with a provided shared set of messages that can be serialized and transferred over different means. The \gls{imc} protocol is defined in a single eXtensible Markup Language (XML) document, which simplify the definition of exiting messages and the creation of new messages. A single XML document ease communication between two node when both node use the same document for message definition. 
\subsection{Dune}
DUNE (DUNE Uniform Navigation Environment) is a runtime environment for unmanned systems on-board software written in C++. DUNE is capable to interact with sensors, payload and actuators, in addition to communication, navigation, control, manoeuvring, plan execution and vehicle supervision. The software separate operations into different task that each has there own thread of execution. DUNE apply a message bus that is responsible for forwarding \gls{imc} message from the producer to all registered receivers, which is the only way different DUNE tasks is communicating. 

A DUNE task is enabled through a configuration file, where the user can choose in which profile the task should be enabled in. The different profile configuration in DUNE allows for testing the same system used in a hardware setting with a simulator.
\subsection{Neptus}
Neptus is a Command and Control software which is used to command and monitor unmanned systems that is written in Java. Neptus is able to provide coherent visual interface to command despite the heterogeneity in the controlled system that it is interacting with.  This allow the operator to command and control unmanned system without the need to dwell into specific command and control software in the unmanned system. The main communication channel for Neptus is \gls{imc}, which makes it interoperable with DUNE or other \gls{imc}- based peer.

Neptus is able to do MRA (Mission Review and Analysis) after a mission is finished. In the MRA phase Neptus analyse the \gls{imc} logs that is collected by e.g. DUNE, such that the result from a completed mission can be presented. In addition Neptus mission review is able to create output files of the log that can be analysed in third party software like Matlab.
\subsection{Glued}
Glued is a minimal Linux operating system distribution, and design with embedded system in mind. It is platform independent, easy to configure and contain only the necessary packages to run on a embedded system. This makes GLUED a light and fast distribution, which is ideal for a on-board operating system for a unmanned system where payload size is normally limited. GLUED is configured through a single configuration file that which can be created for a specific system. A advantage with Glued is that it can be cross-compiled, which allows for compilation of software before it's transferred to the embedded computer.
\section{RTKLIB}\label{ss:Rtklib}
\acrfull{rtklib}\citep{takasu2009development} is a open source program package for standard and precise positioning with \gls{gnss} developed by T. Takasu. \gls{rtklib} can be configured to apply \gls{rtk-gps}, such that raw \gls{gnss} data is used estimate the relative position of the rover with respect to the base station in real time. Figure \ref{figure:RTKLIB_STRUCTURE} shows how \gls{rtklib} can be used in a \gls{rtk-gps} mode. The two main modules here is str2str and rtkrcv. The version of \gls{rtklib} used in this thesis is \gls{rtklib}2.4.2 \citep{Rtklib242}.

Rtklib is configured as a moving baseline, where the baseline between the base station and the rover is accurately estimated with centimeter level accuracy. However the concept of moving baseline indicates that the base station is allowed to move, which require continues calculation of the base station position as a standalone \gls{gps}. Therefore the error sources that are mitigated in the \gls{rtk-gps} solution is present in the \gls{gps} position of the base station. The moving baseline configuration is used since a fixed base station location is not known, and a navigation system that should be used at sea will not have a fixed base station location present.
\begin{figure}[h]
	\centering
		\includegraphics[width=1\textwidth]{figs/RTKLIB.png}
		\caption{The communication structure of \gls{rtklib}}
		\label{figure:RTKLIB_STRUCTURE}
\end{figure}
\newpage

\section{Pixhawk}
3DR Pixhawk is a high-performance autopilot suitable for fixed wing multi rotors, helicopter and other robotic platform that can move. The Pixhawk system comes complete with \gls{gps}, imu, airspeed sensor and magnetometer.
\section{Ardupilot}
Ardupilot is an open-source unmanned aerial vehicle platform, able to control fixed wing \gls{uav} and multicopters. Ardupilot is used for low level control of the \gls{uav}, and is the software that runs on the Pixhawk. Ardupilot is able to communicate to third party software e.g Dune. Ardupilot uses the sensors in the Pixhawk to calculate the position, velocity and attitude of the \gls{uav}, which is sent to DUNE.
\section{JSBsim}
JSBSim \citep{berndt2004jsbsim} is an open-source flight dynamic model that is able to simulate a physical model of an arbitrary aircraft without the need of specific compiled and linked program code. The simulator is design such that a third party software e.g. Ardupilot can expose the model to external forces and moments. This enable \gls{sil} testing of system that is able to run in a hardware configuration with only minor configuration alteration.
\section{X8 and nest payload}
The Skywalker X8 is fixed wing \gls{uav} in a flying wing configuration, which indicate that the \gls{uav} has no tail and clear distinction between the wings and fuselage. The X8 is a popular choice for experimental missions at the \gls{uav}-lab at the Deparment of Engineering Cybernetic since it's durable, cheap and enough space to carry experimental payload. The X8 is used to test the landing path discussed in this thesis, however the navigation system has been tested in both the X8 and a multicopter system.

The hardware configuration used in the X8 and nest systems is based on the proposed hardware in the paper \citep{zolich2015unmanned}. The X8 and the nest systems are installed with a BeagleBone embedded computer with the Glued operating system, which is used to run the Dune system, as well as rtklib. The autopilot used in the X8 is a 3DR Pixhawk with ArduPilot ArduPlane software. For the \gls{rtk-gps} system Ublox Lea M8T GNSS receivers \citep{UbloxDataSheet,UbloxReceiverDescription} are connected to the BeagleBone with uart cable, which is configured with a output rate of 10Hz. The antenna used in the X8 is a Maxtena M1227HCT-A-SMA L1/L2 GPS-GLONASS Active Antenna \citep{maxtena}, and the antenna used in the base station is a Novatel GPS-701-GG \citep{novatel}.

The communication between the X8 and the nest systems is done with Ubiquiti M5 rocket \citep{rocketM5} radios, where the communication between each unit can be done with TCP/UDP/IP.
\cleardoublepage