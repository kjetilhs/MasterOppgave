\chapter{Basis and modelling}
\section{UAV model}
A \gls{uav} model can be described in different coordinate frames depending on which forces and moments are the focus area. The kinematics and kinetics equations \eqref{eq:kinematics}-\eqref{eq:kinetics} used to describe a MAV aircraft is found in \citep{beard2012small}. The kinetic equations is given in the body frame, which is fixed to the frame of the \gls{uav}. The kinematics equations is given as:
\begin{subequations}
\label{eq:kinematics}
\begin{align}\label{eq:kinematicsPosition}
& \begin{bmatrix}
\dot{x} \\
\dot{y} \\
\dot{z}
\end{bmatrix}
=
 R(\Theta)_{Body}^{NED}\begin{bmatrix}
 u \\
 v \\
 w
 \end{bmatrix} \\
& \begin{bmatrix}
\dot{\phi} \\
\dot{\theta} \\
\dot{\psi}
\end{bmatrix}
= 
T(\Theta_{nb})\begin{bmatrix}
p \\
q \\
r
\end{bmatrix}\label{eq:kinematicsAttitude}
\end{align}
\end{subequations}
where $R(\Theta)_{Body}^{NED}$ is the rotation matrix from the body frame to the NED frame, with $\Theta = \begin{bmatrix}
\phi & \theta & \psi
\end{bmatrix}^T.$ The transformation matrix $T(\Theta_{nb})$ is given in \citep{fossen2011handbook} as:
\begin{equation}
T(\Theta_{nb}) = \begin{bmatrix}
1 & \sin(\phi)\tan(\theta) & \cos(\phi)\tan(\theta) \\
0 & \cos(\phi) & -\sin(\phi) \\
0 & \frac{\sin(\phi)}{\cos(\theta)} & \frac{\cos(\phi)}{\cos(\theta)}
\end{bmatrix}
\end{equation}
The kinetic equations is given as:
\begin{subequations}
\label{eq:kinetics}
\begin{align}
\begin{bmatrix}
F_x \\
F_y \\
F_z
\end{bmatrix}
&= R(\Theta)^{Body}_{NED}\begin{bmatrix}
0 \\
0 \\
mg
\end{bmatrix} - \frac{1}{2} \rho V_a^2 S R(\alpha)_{Stability}^{Body}\begin{bmatrix}
F_{Drag} \\
0 \\
F_{Lift}
\end{bmatrix}\\ &+ \frac{1}{2} \rho V_a^2 S \begin{bmatrix}
0 \\
C_y(\beta,p,r,\delta_a,\delta_r) \notag\\
0
\end{bmatrix} + \frac{1}{2} \rho S_{Prop} C_{Prop} \begin{bmatrix}
(K_{Motor}\delta_t)^2-V_a^2 \\
0 \\
0
\end{bmatrix}\\
\begin{bmatrix}
 L \\
 M \\
 N 
 \end{bmatrix} &= \frac{1}{2} \rho V_a^2S\begin{bmatrix}
 C_L(\beta,p,r,\delta_a,\delta_r) \\
 C_M(\alpha,q,\delta_e) \\
 C_N(\beta,p,r,\delta_a,\delta_r)
 \end{bmatrix} + \begin{bmatrix}
 -k_{T_p}(K_\Omega\delta_t)^2 \\
 0 \\
 0
 \end{bmatrix}
\end{align}
\end{subequations}
where $\rho$ is the air density in $kg/m^3$, $S$ is the platform area of the MAV wing, $C_i$ is nondimensional aerodynamic coefficients and $V_a$ is the speed of the MAV through the surrounding air. $\alpha$ and $\beta$ is the attack and side slip angle respectfully. $F_{Drag}$ is the drag force acting on the fuselage, and $F_{Lift}$ is the lift force. $R(\alpha)_{Stability}^{Body}$ is the rotation matrix from the stability frame to the body frame. The stability frame is rotated around the y-axis to the body frame, and is used to express the drag and lift forces. $S_{Prop}$ is the area swept out by the propeller, and $K_{Motor}$,$K_{T_p}$ and $K_\Omega$ is propeller specific constants. The control surface on the MAV is defined into two groups; the wings and the rudder. On the rudder $\delta_e$ controls the elevator deflection and $\delta_r$ the rudder deflection. For the wings $\delta_a$ is the control input from the aileron deflection. The control input for the control input is $\delta_t$.
\section{Landing path modelling}
An autonomous system must be able to create a plan on how the system should move around in the surrounding environment in a feasible way. A minimum requirement for a path is that it is connected. The connection level can be described by the paths smoothness. Parametric continuity is denoted $C^n$ were n is the degree of smoothness. The order of n implies that the n first parametric derivatives match at a common point for two subsequent paths \citep{barsky1989geometric}. Geometric continuity is a relaxed from of parametric continuity in witch discontinuousness in speed is allowed. A table \ref{TB:SmoothnessDescriptions} of geometric and parametric continuity lists the requirement for each smoothness level, which is based definitions presented in \citep{barsky1989geometric}.
Geometric continuity is sufficient for a path following system, which is the main focus of this thesis. Geometric continuity is denoted as $G^n$ were n is the order of continuity.

\begin{table}[H]
\begin{center}
\begin{tabular}{| l | | l |}
\hline
\textbf{Geometrical smoothness level} & \textbf{Description} \\ \hline
$G^0$ & All subpaths are connected \\ \hline
$G^1$ & The path-tangential angle is continuous \\ \hline
$G^2$ & The center of curvature is continuous \\ \hline
\textbf{Parametric smoothness level} & \textbf{Description} \\ \hline
$C^0$ & All subpaths are connected \\ \hline
$C^1$ & The velocity is continuous \\ \hline
$C^2$ & The acceleration is continuous \\ \hline
\end{tabular}
\end{center}
\caption{Smoothness definitions}
\label{TB:SmoothnessDescriptions}
\end{table} 

The definition used for path in this thesis is equation 1.2 in \citep{tsourdos2010cooperative} which state:
\begin{equation}
P_s(x_s,y_s,z_s,\theta_s,\psi_s) \xrightarrow{r(q)} P_f(x_f,y_f,z_f,\theta_f,\psi_f)
\end{equation}
where the subscripts $s$ and $f$ denotes the start pose and finish pose respectfully with $r(q)$ as the path.

\subsection{Straight lines}
The simplest form on path is a straight line between $P_s$ and $P_f$. The straight line is given as 
\begin{subequations}
\begin{align}
& x(s) = a_xs + b_x \\
& y(s) = a_ys + b_y 
\end{align}
\end{subequations}
with $ s \in [0,1] $, where $s$ has not necessary a physical meaning. Then the parametrisation of the straight line is:
\begin{subequations}
\begin{align}
P(0) &= \begin{bmatrix}
x(0) \\
y(0)
\end{bmatrix} = \begin{bmatrix}
b_x \\
b_y
\end{bmatrix} = \begin{bmatrix}
x_s \\
y_s
\end{bmatrix} \\
P(1) &= \begin{bmatrix}
x(1) \\
y(1)
\end{bmatrix} = \begin{bmatrix}
a_x + b_x \\
a_y + b_y
\end{bmatrix} = \begin{bmatrix}
x_f \\
y_f
\end{bmatrix} \rightarrow \begin{bmatrix}
a_x \\
a_y
\end{bmatrix} = \begin{bmatrix}
x_f - b_x \\
y_f - b_y
\end{bmatrix}
\end{align}
\end{subequations}
The tangential vector for a straight line is given as:
\begin{equation}
\psi (s) = \atan2(a_x,a_y)
\end{equation}
A path constructed by straight lines is $G^0$, however since the tangential vector is discontinuous between two line segments with different heading it's not $G^1$[TODO: LEGG INN FIGUR SOM VISER DISKONINUITETEN]. The disadvantage with a path which is $G^0$ is that large discontinuity between two tangential vectors will cause problem for a control system. 
\begin{figure}[H]
\includegraphics[width=0.7\textwidth]{figs/TheoryPlot/StraightLine.eps}
\caption{Straight line path}
\end{figure}
The simplest form of creating a path is a straight line between two way-points. The advantage with a straight line is that it's easy for a guidance system to follow the line, however it will experience a jump in reference when transitioning to another straight line due to discontinuous tangential vector.
\subsection{Dubins path}\label{S:DubinsPath}
An alternative to a straight line path is a path constructed by straight lines and circle. Such a path is Dubins path\citep{dubins1957curves}, which showed that the shortest possible path for a particle that moved with unit speed with maximum curvature would consist of two circles and a straight line. The path consist of two arcs and a straight line. The straight line is tangential to both arcs. A disadvantage with Dubins path is that the curvature is discontinues, which gives a path from $P_s$ to $P_f$ with smoothness level of $G^1$.

A Dubins path that is constructed where the final orientation is fixed has four different ways to be constructed, which is determined by the rotation directions. The four types of Dubins path that is used in this thesis is given in table \ref{Tb:DubinsTurningDirection}.
\begin{table}[H]
\centering
\begin{adjustbox}{max width=\textwidth}
\begin{tabular}{ | l |}
\hline
Right to Right \\
Right to left \\
Left to Right \\
Left to left \\ \hline
\end{tabular}
\end{adjustbox}
\caption{Turning direction for Dubins path with fixed final orientation}
\label{Tb:DubinsTurningDirection}
\end{table}
The equations that is used to construct the path is found in \citep{tsourdos2010cooperative} section 2.2.1, with a constructed path shown in figure \ref{Fig:DubinsPath}. 
\begin{figure}[H]
\def\svgwidth{\textwidth} % Defining the width since Inkscape hasn't done this yet in the .pdf_tex file
\input{InkFig/DubinsPath.pdf_tex}
\caption{Dubins path}
\label{Fig:DubinsPath}
\end{figure}
The first step if to determine the start and final turning circle center. The center is found with the equations:
\begin{subequations}
\begin{align}
& X_{cs} = X_s - R_s\cos(\psi_s \pm \frac{\pi}{2}) \\
& Y_{cs} = Y_s - R_s\sin(\psi_s \pm \frac{\pi}{2}) \\
& X_{cf} = X_f - R_f\cos(\psi_f \pm \frac{\pi}{2}) \\
& Y_{cf} = Y_f - R_f\sin(\psi_f \pm \frac{\pi}{2})
\end{align}
\end{subequations}
where $R_s$ and $R_f$ is the radius of the start and final turning circle respectfully, with $\psi_s$ and $\psi_f$ the start and final heading. The centres for the start and final turning circle is defined as:

\begin{align}
& O_{cs} =
\begin{bmatrix}
X_{cs} \\
Y_{cs}
\end{bmatrix} \\
& O_{cf} =
\begin{bmatrix}
X_{cf} \\
Y_{cf}
\end{bmatrix}
\end{align}
Continuing the centres $O_{cs}$ and $O_{cf}$ is connected with a centreline $c$, where the length is given as:
\begin{equation}
|c| = \sqrt{(X_{cs}-X_{cf})^2+(Y_{cs}-Y_{cf})^2}
\end{equation}
The arc exit and entry point for the start and final circle is calculated by first applying the equations:
\begin{subequations}
\begin{align}
& \alpha = \arcsin\left(\frac{R_f-R_s}{|c|}\right) \\
& \beta = \arctan\left(\frac{Y_{cf}-Y_{cs}}{X_{cf}-X_{cs}}\right)
\end{align}
\end{subequations}
where $\alpha$ is the angle between the length of the center line between the two circles, and the length of the line from the start circle to the exit tangent point. $\beta$ is the angle of the center line with respect to the inertial frame. The exit and entry tangent point is found with the use of table \ref{Tb:ExitEntryTangent}.
\begin{table}[H]\label{Tb:ExitEntryTangent}
\begin{center}
\begin{tabular}{ | l | | l |}
\hline
& \textbf{Turn angle} \\ \hline
$\phi_{right}$ & $\alpha + \beta + \frac{\pi}{2}$ \\
$\phi_{left}$ & $\beta - \alpha + \frac{3\pi}{2}$ \\ \hline
\end{tabular}
\end{center}

\end{table}
With the angle of the exit and entry tangent point the point is given as:
\begin{subequations}
\begin{align}
& x_{P_\chi} = x_{cs} + R_s\cos(\phi) \\
& y_{P_\chi} = x_{cs} + R_s\sin(\phi) \\
& x_{P_N} = x_{cf} + R_f\cos(\phi) \\
& y_{P_N} = x_{cf} + R_f\sin(\phi)
\end{align}
\end{subequations}
The length of the path is calculated in three parts. The first the is the arc length from the start pose to the exit tangent point, then the length of the straight line before the arc length from the entry point to the final pose. The length of the path is given as: 
\begin{equation}
d = R_s\phi_s + d_t + R_f\phi_f
\end{equation}
where $d_t = ||P_N-P_{\chi}||_2$, $\phi_s$ and $\phi_f$ is the arc angle for the start and final circle respectfully.

\section{Position estimation RTK-GPS}\label{ss:rtk-gps}
In \citep{misra2011global} section 7.2.2 \acrfull{rtk-gps} is defined as a rover that receive raw measurements from a reference receiver which is transmitted over a radio link, with a key feature that the rover is able to estimate the integer ambiguities while moving. The reference receiver is usually defined as a base station, and the integer ambiguity is the uncertainty of the number of whole phase cycles between the receiver and a satellite. With the measurements from the base station the rover is able to calculated the distance between itself and the base station, where the distance is referred to as a baseline. The length of the baseline affect the accuracy of the \gls{rtk-gps} solution, due to increased effect of atmospheric disturbance, which is further explain in \ref{Ss:Atmosphere}. However with a short baseline, e.g. $1-2 km$, the atmospheric condition can be considered equal for the base station and the rover, which keeps the solution  at centimetre level accuracy. The concept of \gls{rtk-gps} is depicted in figure \ref{figure:RTK}.
\begin{figure}[H]
	\centering
		\includegraphics[width=0.7\textwidth]{figs/DGPS.png}
		\caption{Concept figure of \acrfull{rtk-gps}}
		\label{figure:RTK}
\end{figure}
The ability for the rover to resolve the integer ambiguity is a key feature in \gls{rtk-gps}. A well used method was purposed in the article \citep{teunissen1994new} which decorrelate the integer ambiguities such that a efficient computation of the least square estimate can be performed. The search method is further explained in \citep{teunissen1995least}. A estimate of the integer ambiguity with sufficient high degree of certainty is referred to as a FIX solution, otherwise the solution is degraded to FLOAT where the integer ambiguity is allowed to be a decimal or a floating point number. When the solution is categorised as FIX the accuracy of the solution is considered on centimetre level, while with a FLOAT solution the accuracy is at a decimetre level. However when a FIX solution is lost, the solution accuracy will not imminently degrade to decimetre level.

In \gls{rtk-gps} the position of the base station must be resolved. This can be achieved by either knowing the position beforehand, which is defined as a kinematic configuration. If the base station position is unknown the \gls{rtk-gps} solver calculates the position on the fly, which is defined as a moving baseline configuration. The unknown is then calculated as a standalone \gls{gnss} receiver, with the accuracy that entails. Therefore the \gls{rtk-gps} system with a moving baseline configuration can never have better global accuracy then what it will get with a single receiver. The advantage with the moving baseline configuration is that \gls{rtk-gps} can be used to find the relative position between two dynamical system using \gls{gnss} in real time. This will be the case in automatic ship landing system, where the base station is on a ship, thus must be allowed to move. The advantage with kinematic mode is that it can give a more accurate position estimate, where the relative position of the rover can be given in either the \gls{ned} or \gls{enu} frame.

\subsection{Error sources}
In order to get high accuracy in the position estimation the different error sources must be identified and removed if possible. This section will identify some of the most significant error sources that can affect the \gls{gnss} signal, and how to remove or mitigate them in the estimation.
\subsubsection{Clock error}
There is drift in both the satellite clock and the receiver clock. The atomic clock in the satellites makes the clock drift negligible from the user perspective. The receiver clock tend to drift, and if not taken into account will cause large deviations in the position estimate from the true position. This error is remove by including a fourth satellite in the position computation. The satellite clock error is given in the satellite message. 

\subsubsection{Ionospheric and tropospheric delays}\label{Ss:Atmosphere}
When the \gls{gps} signals travel though the atmosphere there will be a delay caused by the different atmospheric layers. The atmosphere change the velocity of wave propagation for the radio signal, which results in altered transit time of the signal.
\paragraph{Ionospheric delay}
Gas molecules in the ionosphere becomes ionized by the ultraviolet rays that is emitted by the sun, which release free electrons. These electron can influence electromagnetic wave propagation, such as \gls{gnss} signals. In \citep{vik2014integrated} section 3.5.1 it's stated that the delay caused by the ionosphere usually is in the order of $1-10 $meters. The error can be mitigated by using a double frequency receiver, or by applying a mathematical model to estimate the delay. Both those methods are with a single receiver, however by including a second receiver in a network, e.g. \gls{rtk-gps}, the \gls{gnss} solution system can assume that both receiver receive signal in the same epoch, which means that the signals have experienced the same delay. The rover is then able to remove the error induced from ionospheric disturbance.
\paragraph{Tropospheric delay}
The tropospheric delay is a function of the local temperature, pressure and relative humidity. The effect of tropospheric delay can  vary from $2.4$ meters to $25$ meters depending on the elevation angle of the satellites,\citep{vik2014integrated} section 3.5.1. The error can be mitigated by applying a mathematical model to estimate the tropospheric delay, or by using a elevation mask can remove all satellites with a elevation angle bellow a certain threshold. Similar to ionospheric delay, tropospheric delay can be removed when using two receivers in a network by assuming that the single received by both receivers has experienced the same delay. 
The tropospheric delay is a function of the local temperature, pressure and relative humidity. The effect of tropospheric delay can  vary from $2.4$ meters to $25$ meters depending on the elevation angle of the satellites,\citep{vik2014integrated} section 3.5.1. The error can be mitigated by applying a mathematical model to estimate the tropospheric delay, or by using a elevation mask can remove all satellites with a elevation angle bellow a certain threshold. Similar to ionospheric delay, tropospheric delay can be removed when using two receivers in a network by assuming that the single received by both receivers has experienced the same delay. 

\subsubsection{Multipath}
One of the primary source of error in in a \gls{gnss} receiver is multipath. Multipath happens when the satellite signal is reflected by a nearby surface before if reach the \gls{gnss} antenna. The delay introduced in the signal can make the receiver believe that its position is several meters away form its true position. The easiest way to mitigated multipath is to place the antenna at a location with open skies, with no tall structures nearby. The effect can also be mitigated by choosing a antenna with good multipath rejection capability.

Multipath error uncorrelated between receivers, thus the local receiver must be able to correct for multipath error locally.
\cleardoublepage