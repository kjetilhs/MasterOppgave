%===================================== CHAP 1 =================================

\chapter{Introduction}
Recent development of flying \glspl{uav} has been recognized to provide an attractive alternative to work previously performed by manned operations. Typical work which has attracted attention includes inspection, aerial photography, environmental surveillance and search and rescue. Today \glspl{uav} are mostly operated over land, however in the future this will include over sea as well. This will give some challenges which must be overcome. One of these challenges is that the \gls{uav} need to be able to perform a autonomous landing.

An \gls{uav} can provide an attractive alternative for many maritime operation where today manned aircraft or satellites is the only solution. In the maritime sector \gls{uav} can be used in iceberg management, monitoring of oil spills, search and rescue and maritime traffic monitoring.

An important premise for successful and safe \gls{uav} operation, in particular at sea, is the provision of a robust system for safe landing of the \gls{uav} on a vessel following completed operations. A autonomous landing system require a robust guidance and navigation system, as well as the ability to generate a flyable landing path during flight operation that is within the operation criteria set by the operator. A requirement for the system is that the operator is able to monitor the state of the uav, including the state of the navigation system e.g gps system. If the gps system looses its fixed solution during a critical phase a abort with an evasive manoeuvre might be required. The decision of whether the uav should perform a evasive manoeuvre will be further explored in section (REF:EVASIVE).

Due to regulatory mandate there are restriction on the size of operational area for a uav. Different types of operation is LOS,EVLOS,BVLOS. During LOS the uav must be in line of sight of the operator, which restrict the area were a autonomous landing can take place. In addition there is the risk of loosing satellites during high dynamic behaviour which limits the type of landing path that are available. Therefore a flyable path must be generated for a arbitrary pose, which will provide a gentle landing for the uav.
\section{Background and motivation}
 


The scope of this thesis is the design, implementation and testing of an autonomous landing system for a uav. The main focus in this thesis will be on the navigation and path planning of the landing system. A fellow master student had the main focus of developing the control and guidance system, which will be explain in section ref.


This thesis contain a concept of a robust navigation system for autonomous landing of a UAV, and the implementation of the autonomous landing system. The landing system has been implemented together with a other student, and is a continuation of the master thesis from \citep{Froelich} and \citep{Skulstad&Syversen}. The navigation system that has been implemented apply rtk-gps for position estimation.
\section{Skywalker X8}
The Skywalker X8 is fixed wing \gls{uav} in a flying wing configuration, which indicate that the \gls{uav} has no tail and clear distinction between the wings and fuselage. The X8 is a popular choice for experimental missions at the \gls{uav}-lab at the Deparment of Engineering Cybernetic since it's durable, cheap and enough space to carry experimental payload.
\section{Previous work}
A disadvantage with a net recovery system that is stationary on the deck of a ship is the space requirement for the net, including the safty zone for the personnel required for the uav operation. The paper (Ref multicopter paper when published) addresses this problem by moving the net away from the ship by the means of multirotor uavs. The proposed net recovery system has the advantage that motion induced by the sea is removed, however there is the risk of losing the uav when colliding with the net. A solution that is currently explored is the use of hooks on the uav, which will allow it to grip the net.

A low-cost net recovery system for \gls{uav} with single-frequency \gls{rtk-gps} was described in the paper \citep{skulstad2015net}, which was a result of the work done in the master thesis \citep{Skulstad&Syversen}. The system presented applied RTKLIB together with low-cost single frequency \gls{gps} receivers as navigation system with a customized Ardupilot software. The complete system was able to perform a net landing, however the result showed that further work would require better controllers, and a more robust navigation system.

A continuation of the work done in \citep{Skulstad&Syversen} was done in \citep{Froelich}. The work simulated a autonomous net landing, however no physical experiment was perform. The result in the work indicated that further work on the controllers was required, in addition to the landing path which was not suited for a Visual Line Of Sight  (VLOS) \gls{uav} operation due to no spacial restrictions.
\section{Contributions}
This thesis focus on the navigation system and generation of landing path in the autonomous landing system. The navigation system apply \arcfull{rtk-gps} to provide high accuracy position estimation, which is needed to perform a autonomous landing. The landing path provide a flyable path from any initial position, where the length and direction of the virtual runway is determind by the operator. Through this work, the following contributions has been made:
\begin{itemize}
\item A landing path has been created, which provide a flyable path from any initial position.[INKULDER REF]
\item The landing path has been implemented in the DUNE runtime environment, which is capable to be used in both a stationary and moving net landing. As by product, the plug-in developed by Marcus Frølich \citep{Froelich} was altered and used to specify the parameters used to create the landing path[INKULDER REF]
\item A net nest has been constructed to provide the GPS coordinates for the stationary net, in addition to the heading of the net.[INKULDER REF]
\item A navigation state machine, which is used to control which position solution source should be used in the payload computer. The state machine will try to keep the \gls{rtk-gps} available as long as possible by adding the average difference to the position solution from the Pixhawk for a short duration of time until a new viable \gls{rtk-gps} message is received.[INKULDER REF]
\item  A navigation source interface has been created to provide a visual indicator of which navigation system that is used in the DUNE environment.
\item Finally statistical data on the performance of the X8 during landing during landing has been gather to be used in further work in further developing the landing system.
\end{itemize}
\cleardoublepage