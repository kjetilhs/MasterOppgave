%===================================== CHAP 1 =================================
\chapter{Introduction}
\section{Background and motivation}
Recent development of flying \glspl{uav} has been recognized to provide an attractive alternative to work previously performed by manned operations. Typical work which has attracted attention includes inspection, aerial photography, environmental surveillance and search and rescue. Today \gls{uav} operation are becoming more autonomous, however in order to become fully autonomous a fixed wing \gls{uav} must be able to perform a autonomous landing.

An important premise for successful and safe \gls{uav} operation, is the provision of a robust system for safe landing of the \gls{uav} following completed operations. A autonomous landing system require a path generation system that can create a flyable landing path during flight operation from any initial position. In addition the navigation system must have centimeter level accuracy in order for the \gls{uav} to perform a autonomous landing in a net. However with a accurate navigation system the case of what to do when the positioning system degenerates must be resolved such that system failure does not occur. An other premise is that the position of the net is known, and available for the path system. With a known position of the landing net the \gls{uav} must gracefully perform a graceful decent, preferable a glide slope towards the landing net position. The length of the glide slope will be limited by the operator, which dictates that the \gls{uav} must be in the correct pose before starting the decent.
\section{Previous work}
There has been perform several studies on autonomous landing system, and there currently exist commercial available system. However these are typical expensive, and mostly focused on either military or air traffic industry. An available system for \glspl{uav} is the SkyHook that apply INS/\gls{gps}\citep{SkyHook}, however this system require expensive equipment and is limited to a few \gls{uav} systems. The limitation on type of \gls{uav} and high cost restricts the usage of the recovery system, and motivates the research of a low cost recovery system for fixed wing \gls{uav}.

Studies that has been performed on autonomous landing has mostly focused on vision-based guidance, due to previously limited accuracy in low-cost \gls{gnss} receiver system, which is typically single frequency receivers. In the paper \citep{barber2007autonomous} a landing system was proposed that compared the use of barometric pressure measurement and optic-flow measurement for estimation of height above ground. The landing path composed of a spiral path down to a given altitude where a glide slope was used to guide the MAV down to the landing area. The papers showed that optic-flow measurement reduced the average landing error with several meters, however the technique used to guided the \gls{uav} is not suitable for precision landing due to large average error from target. A low cost recovery system for fixed wing \gls{uav} is proposed in the paper \citep{kim2013fully}, where computer vision is used to find and identify the recovery net. The system was successful in performing a autonomous landing, however it require that the visual image is sent from the \gls{uav} to a ground station. In addition the system require a clear image in order to calculate guidance command for the \gls{uav}, which restricts when the system can used. In the paper \citep{huh2010vision} a vision-based landing system is presented which was successful in performing a automatic landing. The system was aided by a standard IMU and GPS, together with a vision system relaying on color and moments based detection. The system is sensible to lighting condition, however a filtering rule was used to find the landing area. The sensibility to lighting condition is a disadvantage with vision-based guidance system, and therefore it's preferable to create a high accurate positioning system.

A net recovery system for \gls{uav} with single-frequency \gls{rtk-gps} was described in the paper \citep{skulstad2015net}, which was a result of the work done in the master thesis \citep{Skulstad&Syversen}. The system presented applied RTKLIB together with low-cost single frequency \gls{gps} receivers as navigation system with a customized Ardupilot software. The complete system was able to perform a net landing, however the result showed that further work would require better controllers, and a more robust navigation system. A continuation of the work done in \citep{Skulstad&Syversen} was done in \citep{Froelich}. The work simulated a autonomous net landing, however no physical experiment was perform. The result in the work indicated that further work on the controllers was required, in addition to the landing path which was not suited for a Visual Line Of Sight  (VLOS) \gls{uav} operation due to no spacial restrictions.
\section{Contributions}
The objective of this work is to design, implement and test a autonomous landing system for fixed wing \gls{uav}. The focus area for this work has been the design and implementation of a landing plan, and a high accurate navigation system. The landing path generator is a improved version of the landing path used in \citep{Skulstad&Syversen} moved into the DUNE environment, and combined with the landing path generator interface created in the master thesis \citep{Froelich}. The navigation system continues the work started in the master thesis \citep{Spockeli}, where \gls{rtk-gnss} was made available to the DUNE environment. This thesis propose a strategy for handling \gls{rtk-gnss} drop out for a short duration, which is a frequent problem with \gls{rtk-gnss}. The strategy proposed includes fusing data from a secondary \gls{gnss} system in case of loss of \gls{rtk-gnss} lock. This work will not look into the internal algorithm for calculation of \gls{rtk-gnss}, which is performed in the open-source program library RTKlib. The control system used during the testing of the system was developed at the UAVLab by a fellow master student \citep{Sigurd}. The contributions of this thesis are a landing plan generator, a navigation system able to provide high accurate position and velocity information, a redundancy strategy for short loss of \gls{rtk-gnss}, a mobile sensor unit used as a \gls{gps} reference location for net placement and a experimental testing and operational study of the autonomous landing system, summarised as follows:
\begin{enumerate}
\item \textbf{A landing plan generator} has been created, which guaranty a flyable path from any initial position to a landing zone, with a specified decent angle. The landing plan generator has been implemented in the DUNE runtime environment, which is capable to be used in both a stationary and moving net landing. In addition to the implementation of the landing plan a \textbf{\gls{api}} has been created to generate a landing plan.
\item \textbf{A navigation state control system} has been created to manage which positioning system should be used in the DUNE environment. The state machine is designed to switch between the position and velocity information provided by the external navigation system and the \gls{rtk-gnss} position and velocity solution. In addition the \textbf{robustness of the \gls{rtk-gnss}} has been increased by creating a bias estimator, which estimate the bias between \gls{rtk-gnss} and a standalone \gls{gnss} receiver. The bias is used to compensate the external navigation position solution such that the external navigation position solution equals a FIX \gls{rtk-gnss} position solution.
\item \textbf{A navigation source monitor graphical interface} has been created to provide visual indication on the source of the navigation data used in the DUNE environment, in order to create a feedback loop on the state of the navigation system to the operator. The graphical interface is color coded such that changes in the state of the navigation system detected with more ease.
\item \textbf{A mobile sensor with \gls{rtk-gnss}} has been created to be a reference position for a stationary net. The mobile sensor unit can be place on a runway, and used as a reference point for net placement in the command and control monitor, and provide accurate position solution for placement of net through the use of \gls{rtk-gnss}.
\item \textbf{Experimental testing} of the navigation system and landing path generator in the field. The autonomous landing system was tested on the Agdenes airfield with a virtual net placed $25 m$ above the runway. Test result gathered from the field test has been used in a \textbf{operational study} on performance and  feasibility of a autonomous landing at Agdenes airfield.
%\item Finally statistical data on the performance of the X8 during landing during landing has been gather to be used in further work in further developing the landing system.
\end{enumerate}
\section{Outline}
Chapter 2 outlines two path planing strategies which is used in the development of the landing path system. The chapter also contains a model of a \gls{mav}.

Chapter 3 proposes a path and navigation system for a autonomous landing system. The landing path system is separated into two planes, which is the lateral and longitudinal plane. The lateral path is created as a Dubins path, and the longitudinal path is created as a straight line path. The navigation system is controlled by a state machine which controls the source of the positioning and velocity solution, in addition to increasing the robustness of the \gls{rtk-gnss}.

Chapter 4 outlines the software used to create and test the autonomous landing system as well as the hardware configuration used as a basis for the X8 fixed wing \gls{uav}.

Chapter 5 outlines the implementation details of the path and navigation system, including simulation verification of the system.

Chapter 6 present experimental testing of the path and navigation system in the field 

Chapter 7 present the closing discussion with conclusion and recommendation for further work.
\cleardoublepage