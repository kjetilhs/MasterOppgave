%===================================== CHAP 1 =================================
\chapter{Introduction}
\section{Background}
Recent development of flying \glspl{uav} has been recognized to provide an attractive alternative to work previously performed by manned operations. Typical work which has attracted attention includes inspection, aerial photography, environmental surveillance and search and rescue. Today \glspl{uav} are mostly operated over land, however in the future this will include over sea as well. An \gls{uav} can provide an attractive alternative for many maritime operation where today manned aircraft or satellites is the only solution. In the maritime sector \gls{uav} can be used in iceberg management, monitoring of oil spills, search and rescue and maritime traffic monitoring. This will give some challenges which must be overcome. One of these challenges is that the \gls{uav} need to be able to perform a autonomous landing.

An important premise for successful and safe \gls{uav} operation, in particular at sea, is the provision of a robust system for safe landing of the \gls{uav} on a vessel following completed operations. A autonomous landing system require a path generation system that can create a flyable landing path during flight operation from any initial position. In addition the navigation system must have centimeter level accuracy in order for the \gls{uav} to perform a autonomous landing in a net. However with a accurate navigation system the case of what to do when the positioning system degenerates must be resolved such that system failure does not occur. An other premise is that the position of the net is known, and available for the path system. With a known position of the landing net the \gls{uav} must gracefully perform a graceful decent, preferable a glide slope towards the landing net position. The length of the glide slope will be limited by the operator, which dictates that the \gls{uav} must be in the correct pose before starting the decent.
\section{Literature review}
There has been perform several studies on autonomous landing system, and there currently exist commercial available system. However these are typical expensive, and mostly focused on either military or air traffic industry. An available system for \glspl{uav} is the SkyHook that apply INS/\gls{gps}\citep{SkyHook}, however this system require expensive equipment and is limited to a few \gls{uav} systems. The limitation on type of \gls{uav} and high cost restricts the usage of the recovery system, and motivates the research of a low cost recovery system for fixed wing \gls{uav}.

Studies that has been performed on autonomous landing has mostly focused on vision-based guidance, due to previously limited accuracy in low-cost \gls{gnss} receiver system, which is typically single frequency receivers. In the paper \citep{barber2007autonomous} a landing system was proposed that compared the use of barometric pressure measurement and optic-flow measurement for estimation of height above ground. The landing path composed of a spiral path down to a given altitude where a glide slope was used to guide the MAV down to the landing area. The papers showed that optic-flow measurement reduced the average landing error with several meters, however the technique used to guided the \gls{uav} is not suitable for precision landing due to large average error from target. A low cost recovery system for fixed wing \gls{uav} is proposed in the paper \citep{kim2013fully}, where computer vision is used to find and identify the recovery net. The system was successful in performing a autonomous landing, however it require that the visual image is sent from the \gls{uav} to a ground station. In addition the system require a clear image in order to calculate guidance command for the \gls{uav}, which restricts when the system can used. In the paper \citep{huh2010vision} a vision-based landing system is presented which was successful in performing a automatic landing. The system was aided by a standard IMU and GPS, together with a vision system relaying on color and moments based detection. The system is sensible to lighting condition, however a filtering rule was used to find the landing area. The sensibility to lighting condition is a disadvantage with vision-based guidance system, and therefore it's preferable to create a high accurate positioning system.

A net recovery system for \gls{uav} with single-frequency \gls{rtk-gps} was described in the paper \citep{skulstad2015net}, which was a result of the work done in the master thesis \citep{Skulstad&Syversen}. The system presented applied RTKLIB together with low-cost single frequency \gls{gps} receivers as navigation system with a customized Ardupilot software. The complete system was able to perform a net landing, however the result showed that further work would require better controllers, and a more robust navigation system. A continuation of the work done in \citep{Skulstad&Syversen} was done in \citep{Froelich}. The work simulated a autonomous net landing, however no physical experiment was perform. The result in the work indicated that further work on the controllers was required, in addition to the landing path which was not suited for a Visual Line Of Sight  (VLOS) \gls{uav} operation due to no spacial restrictions.
\section{Contributions}
The autonomous landing system must first prove that it can successfully perform a landing on land before it can be tested at sea. The advantage with tests on land is that the net can be assumed stationary, however there are more environmental obstacles that might hinder the \gls{uav}.

This thesis focus on the navigation system and generation of landing path in the autonomous landing system. The navigation system apply \gls{rtk-gps} to provide high accuracy position estimation, which is needed to perform a autonomous landing. The landing path provide a flyable path from any initial position, where the length and direction of the virtual runway is determined by the operator. Through this work, the following contributions has been made:
\begin{enumerate}
\item \textbf{A landing path generator} has been created, which guaranty a flyable path with controlled decent from any initial position. The landing path has been implemented in the DUNE runtime environment, which is capable to be used in both a stationary and moving net landing. In addition to the implementation of the landing path a \textbf{API} has been created to generate a landing path.
\item \textbf{A navigation state control system} has been created to manage which positioning system should be used in the DUNE environment.
\item \textbf{Increased the robustness of the \gls{rtk-gps}} by creating a bias estimator, which estimate the bias between \gls{rtk-gps} and a standalone \gls{gps}.
\item \textbf{A navigation state monitor interface} has been created to provide a visual indicator of which navigation system that is used in the DUNE environment.
\item \textbf{A mobile sensor with \gls{rtk-gps}} has been created to be a reference position for a stationary net.
\item \textbf{Physical experiments} of the navigation system and landing path generator. The experimental testing includes performance of a DUNE guidance system
%\item Finally statistical data on the performance of the X8 during landing during landing has been gather to be used in further work in further developing the landing system.
\end{enumerate}
\section{Outline}
Chapter 2 outlines two path planing strategies which is used in the development of the landing path system. The chapter also contains a model of a \gls{mav}.
Chapter 3 proposes a path and navigation system for a autonomous landing system. The landing path system is separated into two planes, which is the lateral and longitudinal plane. The lateral path is created as a Dubins path, and the longitudinal path is created as a straight line path. The navigation system is controlled by a state machine which manage the 
Chapter 4 explain the software used to create and test the autonomous landing system.
Chapter 5 explain the implementation of the path and navigation system.
Chapter 6 
Chapter 7
\cleardoublepage