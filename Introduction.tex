%===================================== CHAP 1 =================================
\chapter{Introduction}
\section{Background and motivation}
Recent development of flying \glspl{uav} has been recognized to provide an attractive alternative to work previously performed by manned operations. Typical work which has attracted attention includes inspection, aerial photography, environmental surveillance and search and rescue. Today \gls{uav} operation are becoming more autonomous, however in order to become fully autonomous a fixed wing \gls{uav} must be able to perform a autonomous landing.

An important premise for successful and safe \gls{uav} operation, is the provision of a robust system for safe landing of the \gls{uav} following completed operations. A autonomous landing system require a path generation system that can create a flyable landing path during flight operation from any initial position. In addition the navigation system must have centimeter level accuracy in order for the \gls{uav} to perform a autonomous landing in a net. However a accurate navigation system must be able to handle position accuracy degeneration, in order to prevent system failure. Together with a accurate navigation system and path generation system the placement of the net must be known, and made available to the \gls{uav}. With a known position of the landing net the \gls{uav} must gracefully perform a decent, preferable a glide slope towards the landing net position, with a landing path length specified by the operator.
\section{Previous work}
There has been perform several studies on autonomous landing system, and there currently exist commercial available systems. However these are typical expensive, and mostly focused on either military or air traffic industry. An available system for \glspl{uav} is the SkyHook that apply Instrumental Navigation System (INS) with \gls{gnss}\citep{SkyHook}, however this system require expensive equipment and is limited to a few \gls{uav} systems. The limitation on type of \gls{uav} and high cost restricts the usage of the recovery system, and motivates the research of a low cost recovery system for fixed wing \gls{uav}.

Studies that has been performed on autonomous landing has mostly focused on vision-based guidance, due to previously limited accuracy in low-cost \gls{gnss} receiver system, which is typically single frequency receivers. In the paper \citep{barber2007autonomous} a landing system was proposed, which compared the use of barometric pressure measurement and optic-flow measurement for estimation of height above ground. The landing path composed of a spiral path down to a given altitude where a glide slope were used to guide the MAV down to the landing area. The papers showed that optic-flow measurement reduced the average landing error with several meters, however the technique used to guided the \gls{uav} is not suitable for precision landing due to large average error from the landing target. A low cost recovery system for fixed wing \gls{uav} is proposed in the paper \citep{kim2013fully}, where computer vision is used to find and identify the recovery net. The system was successful in performing an autonomous landing, however it require that the visual image is sent from the \gls{uav} to the ground station. In addition the system require a clear image in order to calculate the guidance commands for the \gls{uav}, which restricts when the system can used. In the paper \citep{huh2010vision} a vision-based landing system is presented, which was successful in performing an automatic landing. The system was aided by a standard IMU and GPS, together with a vision system relaying on color and moments based detection. The system is sensible to lighting condition, however an filtering rule was used to find the landing area. The sensibility to lighting condition is a disadvantage with vision-based guidance system, and therefore it's preferable to create an high accurate positioning system.

A net recovery system for \gls{uav} with single-frequency \gls{rtk-gps} was described in the paper \citep{skulstad2015net}, which was a result of the work done in the master thesis \citep{Skulstad&Syversen}. The system presented applied RTKlib together with low-cost single frequency \gls{gps} receivers as navigation system with a customized Ardupilot software. The complete system was able to perform a net landing, however the result showed that further work would require better controllers, and a robust navigation system. An continuation of the work done in \citep{Skulstad&Syversen} was done in \citep{Froelich}. The work simulated an autonomous net landing, however no physical experiment was perform. The work done in \citep{Froelich} moved the autonomous landing system into the DUNE runtime environment, however the autonomous landing system created cannot be used in the field due to the inability of setting spacial restrains on the landing path.

The work done in the master thesis \citep{Spockeli} resulted in the implementation of a \gls{rtk-gnss} system into the DUNE runtime environment, however the \gls{rtk-gnss} system was not integrated with the \gls{uav} navigation system. Other work done at the UAVlab with the goal of creating a autonomous landing system for fixed wing \gls{uav} was performed in the master thesis \citep{Gryte}, where a mathematical model of the skywalker X8 fixed wing \gls{uav} was created. However since the dynamical parameter for the X8 has yet to be determined, deviation in behaviour between simulation and physical flights are expected.
\section{Scope}
The objective of this work is to design, implement and test a autonomous landing system for fixed wing \gls{uav}. The focus area for this work has been the design and implementation of a landing plan, and a high accurate navigation system. The landing plan generator is an improved version of the landing path used in \citep{Skulstad&Syversen} moved into the DUNE environment, combined with the landing plan generator interface created in the master thesis \citep{Froelich}. The navigation system continues the work started in the master thesis \citep{Spockeli}, where \gls{rtk-gnss} navigation data from RTKlib was made available to the DUNE environment.

This thesis propose a path generation system capable of creating a flyable landing plan from any initial position with the guaranty that the \gls{uav} will be able to enter the landing path at the correct height with the correct orientation. The landing plan is composed of a landing path and a approach path where the latter is used to ensure the creation of a flyable path for the \gls{uav} to the start position of the landing path. In addition is a strategy for handling \gls{rtk-gnss} drop out presented, which includes fusing previous valid \gls{rtk-gnss} position solutions together with a external navigation system, to be used as a compensator for the external navigation system during a short duration after an \gls{rtk-gnss} drop out. The compensator is implemented in the DUNE runtime environment, thus avoiding alteration in the \gls{rtk-gnss} system software and ensures that the \gls{uav}s robust \gls{rtk-gnss} navigation system is independent from the \gls{rtk-gnss} system software. The control system tested in this autonomous landing system has been proposed as part of the master thesis \citep{Sigurd}, which has been tested both in a simulator and in the field. The simulator used to verify the autonomous landing system is based on the mathematical model created in the master thesis \citep{Gryte}. For field test a operation study on the execution of a autonomous landing at Agdenes has been performed, in addition to the creation of a mobile sensor unit with \gls{rtk-gnss} to be used as an reference position for stationary net placement.
\section{Contributions}
The contributions of this thesis are a landing plan generator, a navigation system able to provide high accurate position and velocity information, a redundancy strategy for short loss of \gls{rtk-gnss}, a mobile sensor unit used as a \gls{gps} reference location for net placement and a experimental testing and operational study of the autonomous landing system, summarised as follows:
\begin{enumerate}
\item \textbf{A landing plan generator} has been created and implemented in DUNE, which guaranty a flyable path from any initial position to a landing zone, with a specified decent angle. In addition to the implementation of the landing plan an \textbf{\gls{api}} has been created to generate a landing plan.
\item \textbf{A navigation state control system} has been created to manage which positioning system should be used in the DUNE environment. The state machine is designed to switch between the position and velocity information provided by the external navigation system and the \gls{rtk-gnss} system. In addition a \textbf{robust \gls{rtk-gnss} system} has been designed and implemented in DUNE, which includes fusing previous valid \gls{rtk-gnss} position solutions together with a external navigation system, to be used as a compensator for the external navigation system during a short duration after an \gls{rtk-gnss} drop out. With the compensator the \gls{uav} navigation system is able to keep a high accurate positioning system, for a short duration after a \gls{rtk-gnss} drop out has occurred.
\item \textbf{A navigation source monitor} has been created in the command and control software, Neptus, to provide visual indication on the source of the navigation data used in the DUNE environment, in order to give feedback to the operator on the state of the \gls{uav} navigation system. The navigation source monitor based on a color code, in order for the operator to quickly notice alteration in the state of the navigation system.
\item \textbf{A mobile sensor with \gls{rtk-gnss}} has been created, to be used as an reference position for stationary net placement. The mobile sensor unit allows for accurate position solution of the net placement, which is critical for a autonomous stationary net landing system.
\item \textbf{Experimental testing} of the navigation system and landing path generator in the field. The autonomous landing system was tested on Agdenes airfield with a virtual net placed $25 m$ above the runway. Test result gathered from the field test has been used in an \textbf{operational study} on performance and  feasibility of a autonomous landing at Agdenes airfield.
%\item Finally statistical data on the performance of the X8 during landing during landing has been gather to be used in further work in further developing the landing system.
\end{enumerate}
\section{Outline}
Chapter 2 outlines two path planing strategies which is used in the development of the landing path system. The chapter also contains a model of an \gls{mav}.

Chapter 3 presents a path and navigation system for a autonomous landing system. The path system consist of the creation of a landing plan, which contain a landing path and an approach path. The landing path is created as s straight line path relative to the net position and orientation, while the approach path is defined relative to both the landing path and the initial position of the \gls{uav}. The approach path is created as a Dubins path in the lateral plane, and as a straight line path in the longitudinal plane. The navigation system presented consist of a navigation state control system used to integrate a robust \gls{rtk-gnss} system into the \gls{uav} navigation system. The robust \gls{rtk-gnss} consist of the \gls{rtk-gnss} solution from RTKlib and a short \gls{rtk-gnss} loss compensator, which fuse valid \gls{rtk-gnss} position with a external navigation system together to form a compensator that allows the \gls{uav} navigation system to keep a high accurate position solution after a \gls{rtk-gnss} drop out for a short duration.

%Chapter 3 proposes a path and navigation system for a autonomous landing system. The landing path system is separated into two planes, which is the lateral and longitudinal plane. The lateral path is created as a Dubins path, and the longitudinal path is created as a straight line path. The navigation system is controlled by a state machine which controls the source of the positioning and velocity solution, in addition to increasing the robustness of the \gls{rtk-gnss}.

Chapter 4 outlines the software used to create and test the autonomous landing system as well as the hardware configuration used as a basis for the X8 fixed wing \gls{uav}.

Chapter 5 outlines the implementation details of the path and navigation system, including simulation verification of the system. In addition the mobile sensor unit and navigation source monitor are presented.

Chapter 6 present experimental testing of the path and navigation system in the field, with the results used in an operation study of a autonomous landing operation at Agdenes.

Chapter 7 present the closing discussion with conclusion and recommendation for further work.

Appendix A present the complete landing plan generator \gls{api}.

Appendix B present description of the high level control system used in the autonomous landing system.

Appendix C present landing plan parameters used in the SIL simulation and during both test days.

Appendix D present additional results from the performance of the \gls{rtk-gnss} during the field experiment

Appendix E present the configuration file used in RTKlib.

The source code developed in this thesis can be found on the UAV-Lab git server \url{http://uavlab.itk.ntnu.no:88/} under the branch \textit{uavlab}.
\cleardoublepage