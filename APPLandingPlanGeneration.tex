\chapter{Landing plan generation API}\label{AP:APIIMC}
The landing plan generation \gls{api} consist of the \gls{imc} LandingPlanGeneration, where the field with description is given in table 
\begin{table}
\centering
\begin{tabular}{| p{4cm} | p{2cm} | p{6cm} |}
\hline
\textbf{Field name}	& \textbf{Type}	& \textbf{Description} \\ \hline
Command								& Enumerated	& Command the plan database to generate the plan, with the option of executing the plan after generation. \\ \hline
Operation							& Enumerated	& Type of operation started.	\\ \hline
Plan identifier						& Plain text	& Plan identifier.	\\ \hline
Reference latitude					& rad			& Reference latitude for the landing path.	\\ \hline
Reference longitude					& rad			& Reference longitude for the landing path.	\\ \hline
Reference height					& m				& Reference height for the landing path.	\\ \hline
Height over ground					& m				& Offset from the reference height to placement.	\\ \hline
Reference point heading				& rad			& Heading of the reference point in NED frame.	\\ \hline
Distance behind						& m				& Aiming point behind the reference point.	\\ \hline
Final approach length				& m				& The length of the final approach towards the reference point.	\\ \hline
Final approach angle				& rad			& The decent angle of the final approach vector. 	\\ \hline
Glide slope length					& m				& The length of the glide slope in the landing path.	\\ \hline
Glide slope angel					& rad			& The decent angle of the glide slope.	\\ \hline
Approach decent angle				& rad			& The decent angle of the approach path.	\\ \hline
Approach length						& $m$			& The length of the vector from the end of the approach path to the glide slope.	\\ \hline
Landing speed						& $m/s$			& The landing speed.	\\ \hline
Approach speed						& $m/s$			& The approach speed.	\\ \hline
Start turning circle radius			& m				& The radius of the first turning circle.	\\ \hline
Finish turning circle radius		& m				& The radius of the second turning circle.	\\ \hline
Automatic generate landing plan		& Flag			& The approach path will calculate the shortest path form the initial position towards the start of the landing path.	\\ \hline
Start circle turning direction		& Flag			& Manually setting the rotation direction of the first circle.	\\ \hline
Finish circle turning direction		& Flag			& Manually setting the rotation direction of the finish circle.	\\ \hline
Wait at loiter						& Flag			& Setting if after the approach path the \gls{uav} should enter a loiter manoeuvre.	\\ \hline
\end{tabular}
\caption{Table containing the landing plan generation \gls{api}}
\end{table}