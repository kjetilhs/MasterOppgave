\chapter{Background theory}
\section{Position calculation with range measurement}

Position estimation with the use of range measurement apply the pseudo-range measurement, which is defined as the geometrical range in addition to all errors that has affected the measurement. A pseudo-range model can be:

\begin{equation}
y = h(x) + \beta
\end{equation}

where y is the pseudo-range, $h(x) = ||p_O - p||_2$, and $\beta$ is a stochastic disturbance in the measurement.

\section{Estimation}
Different estimation techniques that is used for position estimation. Least squares method is a simple estimation technique with good accuracy for a stationery target. However for a dynamical target a Kalman filter is more suited. Different Kalman filter, which is relevant for target tracking, and sensor fusion practise.

\section{Least square}
Apply the Moore-Penrose psudoinverse
\begin{equation}
\textbf{H}^{\dagger} = (\textbf{H}^T\textbf{W}\textbf{H})^{-1}\textbf{H}^T\textbf{W}
\end{equation}

\section{Kalman filter}
The Kalman filter was presented in the paper (cite kalman) by Rudolf Kalman, and present a filter that in the present of noise. Separate signal from noise.
Different filter that can be used. Present the linear kalman filter, EKF (Partical filter)?

\section{Sensor Fusion}
How to apply different sensors to find position.
\section{Path}
Feasbale path. Order of continuity.
\section{Guidance}
Different strategies that solves the flight approach problem. The problem is to ensure flight path feasibility from every direction.

First apply LOS/ILSO for a traditonal approch. 