\pagestyle{empty}
\begin{abstract}
This thesis present a path and navigation system, which is used in a autonomous net landing system for a fixed wing \gls{uav}. A landing path of a \gls{uav} can be constructed as a straight line path, however in order for a \gls{uav} to follow the landing path it must be in a position from which it has a feasible path to the start position of the landing path. This is not a guaranty during a \gls{uav} operation, which motivates the development of a approach path towards the landing path with the guaranty that the \gls{uav} has a feasible path to the start position of the landing path from any initial start position.

In addition to a path generation system the \gls{uav} require a robust high accurate navigation system. This is accomplished by applying \gls{rtk-gnss}, which can provide centimeter level position accuracy. A shortcoming of the \gls{rtk-gnss} system is that it may loose its lock on satellites leading to loss of functionality. In this work this is compensated for by introducing a secondary \gls{gnss} system. To handle a \gls{rtk-gnss} drop out a robust \gls{rtk-gnss} system is proposed, where previous valid \gls{rtk-gnss} position solutions are fused together with the secondary \gls{gnss} system, to be used as a compensator for the external navigation system. The compensator is designed to enable the external navigation system to achieve the same position accuracy level as the \gls{rtk-gnss} system for a short duration, until the \gls{rtk-gnss} is either reconnected or completely disconnected. With the compensator the \gls{uav} navigation system becomes robust against short drop out of the \gls{rtk-gnss}, and the availability of the \gls{rtk-gnss} is prolonged.

Experimental testing, in addition to \acrfull{sil} verification, of the ability of \gls{uav} to land in a stationary arbitrary placed net has been performed. For improved navigation and performance, an mobile sensor unit have been utilized to provide the required relative position data of the net. This sensor unit was used during testing, where it showed to provide the intended contribution.

%A mobile sensor unit was created to be used as a position reference for net placement, which eased the operational execution of a autonomous landing system. The sensor unit was used then the autonomous landing system was tested in the field at Agdenes airfield. The performance of the autonomous landing system showed that the system is capable of performing a autonomous landing.



%The autonomous landing system is  enabling autonomous landing in a stationary net.
%
%
%
%
%Motivation: Ensuring the uav can enter the landing path at the correct height
%A robust navigation system
%Study on the operational procedure need in order for a autonomous landing operation to be successfully performed
%This report is written as a part of a Master Thesis in 
%
%
%
%Autonomous landing of a fixed wing \acrfull{uav} in a stationary net require a landing plan generator able to create a flyable path towards the net from any initial position, in addition to a accurate robust navigation system. There exits today a
%
%This master thesis present at landing plan generation system and a robust navigation system. 
%
%
%
%Automatic landing of a fixed wing \acrfull{uav} in a net on a ship require an accurate positioning system. There exist today high-end systems with such capability for special applications, e.g military systems and costly commercial systems, which restrict the availability of such systems. To increase the general availability these systems must consist of low-cost components. Here, an alternative is the use of low-cost \gls{gnss} receivers and apply \gls{rtk-gps}, which can provide centimeter level position accuracy. However the processing time for the \gls{rtk-gps} system results in degraded accuracy when exposed to highly dynamical behaviour.
%
%This work present two alternative software and hardware position systems suitable for use in navigation system which apply \gls{rtk-gps}, namely \gls{rtklib} with a Ublox Lea M8T receiver and a Piksi system. Both the Piksi and the Ublox receiver are single-frequency \gls{gnss} receivers. These systems will in this work be compared and their individual capability to provide accurate position estimate will be evaluated. 
%
%The \gls{rtk-gps} system is implemented in DUNE (DUNE:Unified Navigation Environment) framework running on an embedded payload computer on-board an \gls{uav}.
%
%The performance of these position systems are in this work investigated by experimental testing. The testing showed that the \gls{rtklib} performed better than the Piski alternative, and further showed the tested navigation system provide sufficient quality for integration into a control and guidance system, allowing for automatic landing of an \gls{uav} in a net.

\end{abstract}
%\keywords{UAV,RTK-GPS,}