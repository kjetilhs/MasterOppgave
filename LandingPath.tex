\part{Method}

\chapter{Landing path}
The path generator in the  autonomous landing system is designed to enable \gls{uav} landing in both a stationary or moving net. The path is created in two main stages. The first is the creation of the virtual runway, which is defined as a straight line along the heading of the net. The second stage apply a lateral Dubins path \ref{S:DubinsPath} and longitudinal straight line path to create a path that ensures that the \gls{uav} is able to enter the straight line along the virtual runway at the correct height and attitude. The design is inspired by the work done in \citep{Skulstad&Syversen} were way-points was used to guide the \gls{uav} toward the landing approach.
\section{The virtual runway}\label{SS:netApproach}
The virtual runway is inspired by the work done in \citep{Skulstad&Syversen} where waypoint was used to create a straight line path towards the net. This method proved successful, and since the \gls{uav} descent towards the net should be as controlled as possible only small angles is used when transitioning between way-points. The straight line path is constructed relative to the net as shown in figure \ref{Fig:LandingPhase}, with way-points given as:
\begin{subequations}
\begin{align}
&\mathbf{WP4} = 
\begin{bmatrix}
-a0 \\
0 \\
h_{nc} + a1\tan(\gamma_a) 
\end{bmatrix}\\
&\mathbf{WP3} = 
\begin{bmatrix}
a1 \\
0 \\
h_{nc} - a1\tan(\gamma_a)
\end{bmatrix}\\
&\mathbf{WP2} = \mathbf{WP3} + 
\begin{bmatrix}
a2 \\
0 \\
a2\tan(\gamma_d)
\end{bmatrix}\\
&\mathbf{WP1} = \mathbf{WP2} + 
\begin{bmatrix}
a3 \\
0 \\
0 \\
\end{bmatrix}
\end{align}
\end{subequations}
were the description of the parameters used is given in table \ref{Tb:Approach Parameters}. The net is placed between the fourth and third way points such that transitional behaviour do not occur during the finale stage of the net landing. In addition the path has been made with the assumption that the $\gamma_a$ and $\gamma_d$ is considered small. This assumption is made to ease the demand of the controllers used in the landing system.
\begin{table}[H]
\begin{center}
    \begin{tabular}{ | l | l |}
    \hline
    \textbf{Parameter} & \textbf{Description} \\ \hline
    $a0$ & The distance behind the net \\ \hline
    $a1$ & The distance in front of the net \\ \hline
    $a2$ & The length of the glide slope \\ \hline
    $a3$ & The length of the approach towards the glide slope \\ \hline
    $\gamma_a$ & The net attack angle \\ \hline
    $\gamma_d$ & The glide slope angle \\ \hline
    \end{tabular}
\end{center}
\caption{Net approach parameters }
\label{Tb:Approach Parameters}
\end{table}
The way point vectors are rotated into the NED frame by a rotation around the z-axes.
\begin{equation}
WP^n = R(\psi_{net})WP^b
\end{equation}
were $\psi_{net}$ is the heading of the net, and $R(\psi_{net})$ is the rotation matrix around the z-axis.
\begin{figure}\label{Fig:LandingPhase}
\def\svgwidth{\textwidth} % Defining the width since Inkscape hasn't done this yet in the .pdf_tex file
\input{InkFig/LandingPhase.pdf_tex}
\end{figure}

\section{The landing approach}\label{SS:LandingApproach}
In order to ensure that the \gls{uav} follows the path along the virtual runway it must be at the correct height with the correct attitude from any initial position. In addition it's desirable the the decent angle is kept small to ease the strain on the control system.

The landing approach consist of a Dubins path in the lateral plane and a straight line path in longitudinal plane.

\subsection{Lateral path}
Dubins path was chosen due to it's simplicity in description and it fulfils the requirement that it's smooth enough for path following.
\subsection{Longitudinal path}


The lateral path was chosen for the autonomous landing system is Dubins path, due to it's simplicity in description and it fulfils the requirement that it's smooth enough for path following. Since it's desired for the \gls{uav} to decent with small decent angle the longitudinal path is constructed with straight lines. Combined with Dubins path, the circles in the start and end of the path becomes spirals. This increase the demands on the controllers used, since they has to both control the heading in addition to the decent rate with the same control surface. This is the main reason for small angles in the longitudinal path.

The path generation system is designed such that the operator can choose how the path should be generated. The system allows for manual creating of the path, or simply create the shortest path from the start pose to the end pose, which is the first way-point in the path towards the net.